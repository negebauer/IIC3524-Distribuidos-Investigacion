\documentclass[10pt]{extarticle}

%Paquetes utilizados en esta tarea
\usepackage{fullpage}
\usepackage[utf8]{inputenc}
\usepackage[spanish]{babel}
\usepackage{epsfig}
\usepackage{amsmath}
\usepackage{amssymb}
\usepackage{epstopdf}
\usepackage[hidelinks]{hyperref}
\usepackage{xcolor}
\usepackage{algorithmic}
\usepackage[nothing]{algorithm}
\usepackage{listings}
\usepackage{color}

\definecolor{codegreen}{rgb}{0,0.6,0}
\definecolor{codepurple}{rgb}{0.58,0,0.82}
\definecolor{backcolour}{rgb}{0.95,0.95,0.92}

\lstdefinestyle{mystyle}{
    backgroundcolor=\color{backcolour},
    commentstyle=\color{codegreen},
    keywordstyle=\color{magenta},
    stringstyle=\color{codepurple},
    basicstyle=\footnotesize,
    breakatwhitespace=false,
    breaklines=true,
    captionpos=b,
    keepspaces=true,
    showstringspaces=false,
    showtabs=false,
    tabsize=2
}

\lstset{style=mystyle}

%Definiciones de comandos, para reutilizar secuencias frecuentes o ahorrar
%código
\newcommand{\RR}{\mathbb{R}}
\newcommand{\lb}{\\~\\}
\newcommand{\la}{\leftarrow}

\newcommand{\twopartdef}[4]
{
	\left\{
		\begin{array}{ll}
			#1 &  \text{si }#2 \\
			#3 &  \text{si }#4
		\end{array}
	\right.
}

\newcommand{\threepartdef}[6]
{
	\left\{
		\begin{array}{ll}
			#1 &  \text{si }#2 \\
			#3 &  \text{si }#4 \\
			#5 &  \text{si }#6
		\end{array}
	\right.
}

\makeatletter

\makeatother

\begin{document}

\begin{tabular}{ccl}
 \begin{tabular}{c}
 \includegraphics[width=2.5cm]{imgs/logo.pdf}
\end{tabular}
&\ \ \ &
\begin{tabular}{l}
 PONTIFICIA UNIVERSIDAD CATÓLICA DE CHILE               \\
 DEPARTAMENTO DE CIENCIA DE LA COMPUTACIÓN              \\
 IIC3524 {-} Tópicos avanzados de sistemas distribuidos \\
\end{tabular}
\end{tabular}

\begin{center}
 \bf {\Huge Chapel}

 \vspace{0.2cm}
 \bf {\Large Propuesta investigación HPC}

 \vspace{0.2cm}
 \bf 4 de julio de 2017

 \vspace{0.2cm}
 \bf Nicolás Gebauer {-} 13634941

 \vspace{0.2cm}
 \bf \href{https://github.com/negebauer}{\color{blue!60}@negebauer} {-} \href{https://github.com/negebauer/IIC3524-Investigacion}{\color{blue!60}repo Investigación}
 \noindent\rule{16cm}{0.05pt}
\end{center}

\subsection*{Problema}
Cada día se tienen más datos disponibles con los cuales uno puede realizar cómputos interesantes. Cada día se tiene computadores más poderosos y juntando varios de ellos se tienen \textit{clusters}. En este contexto se hace importante poder aprovechar estas plataformas \textit{HPC}.\\
Hoy esto es posible con lenguajes como \textit{C} y \textit{frameworks} adicionales como \textit{OpenMP} y \textit{MPI} que permiten aprovechar los recursos disponibles para hacer cómputos paralelos.\\

\textit{OpenMP} permite trabajar en distintas tarea paralelas, aprovechando las únidades de cómputo que tiene disponible la máquina que se utiliza.\\
\textit{MPI} permite además de trabajar usando los varios \textit{cores} disponibles en una máquina utilizar distintos nodos, es decir, realizar cómputos en varios computadores al mismo tiempo de manera paralela. Esto permite, por ejemplo, trabajar con grandes \textit{sets} de datos para obtener resultados.\\
Otras herramientas como \textit{CUDA} y \textit{OpenCl} permiten usar más recursos de los sistemas, como sus tarjetas \textit{GPU} para acelerar los cómputos.\\

Ante tantas herramientas, el campo del desarrollo de programas \textit{HPC} tiene una barrera de entrada muy grande. Primero se necesita ser compentente en lenguajes como \textit{C} que no son lenguajes modernos, es decir, no tienen una alta prioridad en el aprendizaje de los programadoresde hoy. Luego uno debe aprender sobre las distintas herramientas para poder aprovecharlas.

% \pagebreak
\subsection*{Solución}
Para poder atacar esta problemática la empresa \href{http://www.cray.com}{\color{blue!60}Cray} decidió crear un nuevo lenguaje, llamado \textit{Chapel}. El objetivo principal de \textit{Chapel} es acercar la programación orientada a \textit{HPC} a más desarrolladores. Algunos de los objetivos de \textit{Chapel} que es importante destacar.

\begin{itemize}
 \item Un única herramienta para \textit{HPC}
 \item Lenguaje alto y bajo nivel
 \item Lenguaje moderno
\end{itemize}

El primer punto hace referencia a que el lenguaje engloba muchas formas de hacer trabajos paralelos en varios nodos que actualmente requieren distintas herramientas. Esto facilita el desarrollo ya que solo es necesario aprender a utilizar un nuevo lenguaje.\\

El segundo destaca la importancia de que el lenguaje facilita el comenzar a realizar trabajos en paralelos a través de directivas de alto nivel. Esto reduce la barrera de entrada para quienes son nuevos en \textit{HPC}. A la vez permite un manejo a bajo nivel, lo que le permite a quienes tienen conocimientos más avanzados poder trabajar más cercano a la máquina, por ejemplo, optimizando de mejor forma el uso de memoria.

Como ejemplo, en \textit{Chapel} correr un proceso en varios nodos es tan simple como ejecutar:

\begin{lstlisting}[language=Chapel]
// Un task para cada local (nodo) disponible
coforall loc in Locales {
	// Movemos el computo al local
	on loc {
		// Imprimimos un mensaje
		writeln('Hola desde local ', here.id);
			// here hace referencia al local que esta corriendo el codigo
	}
}
\end{lstlisting}

El tercer punto se refiere a que el lenguaje fue escrito tomando inspiración de lenguajes más moderno como \textit{Python} y \textit{Java}, lo que hace que nuevos desarrolladores se sientan más cómodos al interactuar con el. También tomaron inspiración de \textit{C} para que desarrolladores veteranos no queden excluidos. Por ejemplo, se puede programar con \textit{OOP} si uno lo desea.

% \pagebreak
\subsection*{Desarrollo}
El objetivo de este trabajo es, en pocas palabras, hacer una prueba del lenguaje \textit{Chapel}. Para ello se aprovecha la \href{https://github.com/negebauer/IIC3524-T2}{\color{blue!60}Tarea 2} que se realizó en el curso con \textit{MPI}. El código resultante puede verse en el siguiente \href{https://github.com/negebauer/IIC3524-Investigacion}{\color{blue!60}repositorio de github}, Junto con instrucciones sobre como activar chapel, configurar algunas variables de entorno (como los \textit{hosts} a utilizar), el comando \verb|chpl| para compilar programas de chapel y como correrlos.

% \pagebreak
\subsection*{Referencias}
Chapel chapter, Bradford L. Chamberlain, \textit{Programming Models for Parallel Computing}, edited by Pavan Balaji, published by MIT Press, November 2015. Disponible en \href{http://chapel.cray.com/publications/PMfPC-Chapel.pdf}{\color{blue!60}http://chapel.cray.com/publications/PMfPC-Chapel.pdf}

%\lstinputlisting[language=Python, firstline=2, lastline=12]{../code/main.chpl}

\end{document}
