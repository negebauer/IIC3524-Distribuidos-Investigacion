\documentclass[10pt]{extarticle}

%Paquetes utilizados en esta tarea
\usepackage{fullpage}
\usepackage[utf8]{inputenc}
\usepackage[spanish]{babel}
\usepackage{epsfig}
\usepackage{amsmath}
\usepackage{amssymb}
\usepackage{epstopdf}
\usepackage[hidelinks]{hyperref}
\usepackage{xcolor}
\usepackage{algorithmic}
\usepackage[nothing]{algorithm}
\usepackage{listings}
\usepackage{color}

\definecolor{codegreen}{rgb}{0,0.6,0}
\definecolor{codepurple}{rgb}{0.58,0,0.82}
\definecolor{backcolour}{rgb}{0.95,0.95,0.92}

\lstdefinestyle{mystyle}{
    backgroundcolor=\color{backcolour},
    commentstyle=\color{codegreen},
    keywordstyle=\color{magenta},
    stringstyle=\color{codepurple},
    basicstyle=\footnotesize,
    breakatwhitespace=false,
    breaklines=true,
    captionpos=b,
    keepspaces=true,
    showstringspaces=false,
    showtabs=false,
    tabsize=2
}

\lstset{style=mystyle}

%Definiciones de comandos, para reutilizar secuencias frecuentes o ahorrar
%código
\newcommand{\RR}{\mathbb{R}}
\newcommand{\lb}{\\~\\}
\newcommand{\la}{\leftarrow}

\newcommand{\twopartdef}[4]
{
	\left\{
		\begin{array}{ll}
			#1 &  \text{si }#2 \\
			#3 &  \text{si }#4
		\end{array}
	\right.
}

\newcommand{\threepartdef}[6]
{
	\left\{
		\begin{array}{ll}
			#1 &  \text{si }#2 \\
			#3 &  \text{si }#4 \\
			#5 &  \text{si }#6
		\end{array}
	\right.
}

\makeatletter

\makeatother

\begin{document}

\begin{tabular}{ccl}
 \begin{tabular}{c}
 \includegraphics[width=2.5cm]{imgs/logo.pdf}
\end{tabular}
&\ \ \ &
\begin{tabular}{l}
 PONTIFICIA UNIVERSIDAD CATÓLICA DE CHILE               \\
 DEPARTAMENTO DE CIENCIA DE LA COMPUTACIÓN              \\
 IIC3524 {-} Tópicos avanzados de sistemas distribuidos \\
\end{tabular}
\end{tabular}

\begin{center}
 \bf {\Huge Chapel}

 \vspace{0.2cm}
 \bf {\Large Propuesta investigación HPC}

 \vspace{0.2cm}
 \bf 23 de junio de 2017

 \vspace{0.2cm}
 \bf Nicolás Gebauer {-} 13634941

 \vspace{0.2cm}
 \bf \href{https://github.com/negebauer}{\color{blue!60} @negebauer} {-} \href{https://github.com/negebauer/IIC3524-Investigacion}{\color{blue!60}repo Investigación}
 \noindent\rule{16cm}{0.05pt}
\end{center}

\subsection*{Introducción}
Lorem ipsum

\begin{lstlisting}[language=Python]
import numpy as np

def incmatrix(genl1,genl2):
    m = len(genl1)
    n = len(genl2)
    M = None #to become the incidence matrix
    VT = np.zeros((n*m,1), int)  #dummy variable

    #compute the bitwise xor matrix
    M1 = bitxormatrix(genl1)
    M2 = np.triu(bitxormatrix(genl2),1)

    for i in range(m-1):
        for j in range(i+1, m):
            [r,c] = np.where(M2 == M1[i,j])
            for k in range(len(r)):
                VT[(i)*n + r[k]] = 1;
                VT[(i)*n + c[k]] = 1;
                VT[(j)*n + r[k]] = 1;
                VT[(j)*n + c[k]] = 1;

                if M is None:
                    M = np.copy(VT)
                else:
                    M = np.concatenate((M, VT), 1)

                VT = np.zeros((n*m,1), int)

    return M
\end{lstlisting}

\lstinputlisting[language=Python, firstline=2, lastline=12]{../code/main.chpl}

\end{document}
